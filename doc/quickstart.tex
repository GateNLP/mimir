This chapter is aimed at the impatient reader who wants a working system as
quickly as possible. The technical detail is deliberately kept at a minimum so,
while you will hopefully end up with something that works, you will not
necessarily understand how it all fits together. For that, please read the
remainder of this guide.

\section{Set Up Your Environment}
We suggest you try this on a 64~bit operating system, as that is better suited
for running \Mimir{}. A 32~bit system would also work, but the maximum sizes for
the indexes would be limited.

In order to build and run a \Mimir{} server you will need the following pieces
of software installed on your system:
\begin{description}
  \item[Java Development Kit] version 7 or newer. If you don't have one, you can
  download one from 
  Oracle\footnote{\url{http://www.oracle.com/technetwork/java/javase/downloads/index.html}}.
  Make sure you get the JDK and not the Java Runtime Environment (JRE), as that
  would not be suitable. Once installed, make sure your \verb!JAVA_HOME!
  environment variable points to the location where the JDK was installed.  Make
  sure that the \verb!$JAVA_HOME/bin! location is on your \verb!PATH!.
  \item[Apache ANT] version 1.8.1 or later. You can download it from
  \url{http://ant.apache.org/}. Once installed, make sure your \verb!ANT_HOME!
  environment variable points to the top-level directory of your installation.
  Make  sure that the \verb!$ANT_HOME/bin! location is on your \verb!PATH!.
  \item[Grails] version 2.5.4. You can download this from
  \url{http://grails.org}. Once installed, make sure your \verb!GRAILS_HOME!
  environment variable points to the top-level directory of your installation.
  Make sure that the \verb!$GRAILS_HOME/bin! location is on your \verb!PATH!.
  \textbf{Note that \Mimir\ \mimirversion\ requires Grails 2.5.4, it will not
  work with 3 or later}. Earlier versions of \Mimir\ used different versions
  of Grails, make sure you are reading the documentation for the specific
  version you are trying to run.
  \item[Working Internet Connection] The next step, described below, is the
  building of the \Mimir{} library. This starts by automatically downloading all
  the required dependencies, so it requires a working Internet connection. Once
  the software is built, it can work without an remote connection.
  \item[GATE Developer] \Mimir{} is an indexer for GATE Documents. The simplest
  way of generating some GATE documents to be indexed is by using the GATE
  Developer tool\footnote{GATE Developer is available at
  \url{http://gate.ac.uk/download/}. Usage of GATE Developer is beyond the scope
  of this document, so we assume you have a basic understanding of how to use
  it. If not, a good place to start is the tutorials page at
  \url{http://gate.ac.uk/demos/developer-videos/}.}. 
\end{description}
%
\section{Build and Run a \Mimir{} Web Application}
%
After all the prerequisites are installed, we can move to building a \Mimir{}
application. For the purposes of this demo, we will build the {\tt mimir-cloud}
application, which is included in the source tree.

The following steps will help you build the {\tt mimir-cloud} application.
Commands that you have to execute are formatted in a distinctive font
\cmd{like this}.
\begin{enumerate}
  \item {\bf Download the \Mimir{} sources}, if you do not already have a copy.
  You can get either an archive of the entire source tree, or check it out
  directly from our subversion repository. Instructions for doing so are
  available on \Mimir{}'s web page at:
  \url{http://gate.ac.uk/mimir/index.html}.
  If you downloaded the .tar.gz archive on Windows we recommend not using the
  popular Winzip utility, as that sometimes mangles the file names. 
  7-Zip\footnote{\url{http://www.7-zip.org/}} and the Cygwin ``tar'' utility are
  known to work correctly in this respect, and other free archiving tools are 
  available that support the {\tt .tar.gz} format.
  Unpacking a source archive (or checking out the source code with subversion)
  will create a new directory called {\tt mimir} containing all the source
  files.
  \item {\bf Build \Mimir{}:} change to the top level directory where you
  unpacked the downloaded \Mimir{} sources. If you can see the {\tt mimir-core},
  {\tt mimir-client}, {\em etc.} directories, then you are in the correct
  directory. Execute the \cmd{ant} command. This will download all the
  required dependencies, compile all the \Mimir{} libraries, and build the {\tt
  mimir-web} Grails plugin.
  
  If you have multiple Grails versions installed, and Grails 2.5.4 is not the
  default, you must give priority to Grails 2.5.4.  Do so by executing
  \cmd{export GRAILS\_HOME=/path/to/grails-2.5.4}, and then use the following:
  \cmd{ant -Dgrails.bin=\$GRAILS\_HOME/bin/grails} (instead of \cmd{ant}) to
  override the default Grails settings.
  \item {\bf Run the mimir-cloud application:}  change to the {\tt
  mimir-cloud} directory (\cmd{cd mimir-cloud}) and execute the \cmd{grails prod
  run-app} command. This will start the application and will notify you which
  URL you should use in your browser to access it (normally
  \url{http://localhost:8080/mimir-cloud/}).
\end{enumerate}
%
\section{Create, Populate, and Search an Index}
%
\begin{enumerate}
\setcounter{enumi}{3}
  \item {\bf Set-up your new \Mimir{} application:}
  navigate to the administration page. You will be prompted to configure your
  \Mimir{} instance. After clicking the link, enter the path to a local writable directory
  where new indexes will be created, and click the {\em Update}.button.
  \item \label{step:create} {\bf Create a new index:} navigate back to the
  administration page (by clicking the link at the top of the page). Under the {\em Local Indexes}
  section, click the {\em create a new local index} link. Give it a name (e.g.
  {\tt `test'}), and click the {\em create} button. Back on the administration
  page, click the name of the newly created index. This will take you to the
  index details page, where you can find the {\em Index URL} attribute. Make a
  note of its value, as you will need it later.
  \item {\bf Populate the new index:} 
  \begin{enumerate}
    \item Start GATE Developer, load the ANNIE application (Main Menu 
    $\rightarrow$ File $\rightarrow$ Load ANNIE System $\rightarrow$ with
    Defaults).
    \item Open the CREOLE Plugin Manager ((Main Menu $\rightarrow$ File
    $\rightarrow$ Manage CREOLE Plugins), and add a new plugin directory
    pointing at the {\tt mimir-client} directory inside the \Mimir{}
    distribution. Make sure the new plugin is loaded by checking the
    appropriate check-box.
    \item Load a new instance of {\em \Mimir{} Indexing PR} (Main Menu
    $\rightarrow$ File $\rightarrow$ New Processing Resource $\rightarrow$
    \Mimir{} Indexing PR), and add it to the end of the ANNIE application.
    \item Make sure that the {\tt mimirIndexUrl} parameter for the new PR is set
    to the {\em Index URL} value obtained at Step~\ref{step:create}.
    \item Load some test documents (e.g. some web pages from news web sites),
    create a GATE Corpus, add all the documents to the corpus, and set the
    newly corpus as the target for the ANNIE application.
    \item Run the ANNIE application. This will annotate the documents created
    during the previous step. The \Mimir{} Indexing PR instance will make
    sure the annotated documents are sent for indexing to your new Local Index.
  \end{enumerate}
  \item {\bf Search the new index:} as soon as the index has started indexing
  document, you can used it to search by clicking the {\em search} link next to
  the name of your new index. There is a time delay between documents being
  submitted for indexing and them being available for searching. YOu can speed
  this process up by manualy performing a {\em sync-to-disk} operation or by
  reducing the time interval between batches. Both of these actions are
  available on the index administration page.
\end{enumerate}

To shut down the running web application, create a file named {\tt
.kill-run-app} in the {\em mimir-cloud} directory, and wait for the application
to shut itself down. If that does not work (creating files with `.' at the start
of their names is sometimes difficult on Windows), then you can just focus the
command prompt window where you started the application and interrupt it by
pressing the {\tt Ctrl-C} key combination. This might, on rare occasions,
invalidate the database of the \Mimir{} web application, but it would not affect
any indexes you have created (they would simply disappear from the list and
you would need to re-import them).

To deploy the \Mimir{} web application to an application server (such as Apache
Tomcat) run the \cmd{grails prod war} command in the mimir-cloud directory. A
{\tt mimir-cloud-\{version\}.war} file will be created for you in the {\tt
target} sub-directory.
